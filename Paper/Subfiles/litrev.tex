\onlyinsubfile{\setcounter{section}{1}}
\section{Literature Review}\notinsubfile{\label{sec:litrev}}

\par

\subsection{Trust}

\par There are three strands of the trust literature of interest to this paper. First, this paper draws from the literature regarding the detereminants of trust. An example of this is \cite{Alesina2000}. Key takeaway is that the standard demographic control variables (age, race, education, gender) as well as living in a community with much heterogeneity in terms of either race or income.

%%%%%%%%%%%%%%%%%%%%%%%%%%

\par This paper also seeks to contribute to the trust literature regarding the empirical relationship between measures of trust and economic performance. A notable example relevant for this paper is \cite{lgpslz2008} which finds that trust can lead to greater stock market participation. %%%what my paper does differently%%%

\par A related work \cite{Guiso2004} finds that measures of social capital and trust leads to greater use of financial instruments like checks, as well as poortfolio diversification away from holding cash and towards holding stocks and other assets. %%%what my paper does differently%%%

\par A paper which serves as a key motivation for this work is \cite{jbpglg2016}. They find that the relationship between trust and income is humped shaped; an intermediate level of trust is associated with the maximal level of income. %%%what my paper does differently%%%

%%%%%%%%%%%%%%%%%%%%%%%%%%%%%%%%5

\par Causal effects in the trust literature also inform this paper. \cite{Alsan2018} find that the 1972 public revelation of the Tuskegee Syphilis Study lead to  lower trust in black males in medical institutions, ultimately leading to worse health outcomes over a lifetime.    %%%what my paper does differently%%%

\par \cite{Algan2010} find that aggregate trust has a large positive causal effect on GDP per capita by showing that a measure of inherited trust by second generation immigrants in the U.S. is strongly correlated with aggregate trust in the origin country (and is persistent across generations). %%%what my paper does differently%%%

 
\subsection{Returns}

\par An important work in the literature on empirical estimates of the rate of return is \cite{aflgdmlp20} which characterizes the persistent component of returns for indivdiuals in Nowegian population data. This work also provides estimates of the distribution of returns within narrowly defined asset class, as well as for net wealth, which can serve as a useful benchmark for comparison. %%%what my paper does differently%%%

\par \cite{Daminato2024} also documents the persistent component of returns in the PSID, and then use the data to calibrate an return-earnings process within a life cycle model of consumption-saving behavior. %%%what my paper does differently%%%