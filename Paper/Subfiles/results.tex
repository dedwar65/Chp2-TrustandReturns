\onlyinsubfile{\setcounter{section}{3}}
\section{Results}\notinsubfile{\label{sec:results}}

%\setcounter{page}{0}\pagenumbering{arabic}
\subsection{Determinants of trust}

\par First, I go beyond correlations between the variables by trying to consider which variables explain the variation in the trust measures.

\begin{table}[htbp]\centering
\def\sym#1{\ifmmode^{#1}\else\(^{#1}\)\fi}
\caption{General trust (2020) on controls}
\label{tab:trust_reg_general}
\begin{tabular*}{0.85\hsize}{@{\hskip\tabcolsep\extracolsep\fill}l*{2}{D{.}{.}{-1}}}
\toprule
                    &\multicolumn{1}{c}{Demographics}&\multicolumn{1}{c}{Full controls}\\
\midrule
Female              &        0.19         &        0.27         \\
                    &      (0.17)         &      (0.18)         \\
Years of education  &        0.05\sym{*}  &        0.04         \\
                    &      (0.03)         &      (0.03)         \\
Married             &        0.50\sym{***}&        0.17         \\
                    &      (0.18)         &      (0.19)         \\
NH Black            &       -0.99\sym{***}&       -1.07\sym{***}\\
                    &      (0.22)         &      (0.23)         \\
Hispanic            &       -0.38         &       -0.34         \\
                    &      (0.27)         &      (0.27)         \\
NH Other            &       -0.19         &       -0.19         \\
                    &      (0.32)         &      (0.33)         \\
Depression          &                     &       -0.14\sym{***}\\
                    &                     &      (0.05)         \\
Health conditions   &                     &       -0.05         \\
                    &                     &      (0.07)         \\
Covered by Medicare &                     &        0.15         \\
                    &                     &      (0.30)         \\
Covered by Medicaid &                     &       -0.58\sym{**} \\
                    &                     &      (0.28)         \\
Has life insurance  &                     &        0.25         \\
                    &                     &      (0.18)         \\
Number of reported divorces&                     &       -0.16         \\
                    &                     &      (0.11)         \\
Number of reported times being widowed&                     &        0.00         \\
                    &                     &      (0.25)         \\
Constant            &        2.04         &        2.90         \\
                    &      (1.49)         &      (1.84)         \\
\midrule
Observations        &      894.00         &      875.00         \\
Adj. R-squared      &        0.09         &        0.12         \\
\bottomrule
\multicolumn{3}{l}{\footnotesize Standard errors in parentheses}\\
\multicolumn{3}{l}{\footnotesize Robust standard errors in parentheses. Trust and controls from 2020. Age bins (5-yr) included; coefficients omitted.}\\
\multicolumn{3}{l}{\footnotesize \sym{*} \(p<0.10\), \sym{**} \(p<0.05\), \sym{***} \(p<0.01\)}\\
\end{tabular*}
\end{table}
  

\subsection{Income and trust}

\par After understanding the determinants of trust in the HRS data, I wanted to see if the \say{hump-shape} relationship between trust and income found by \cite{jbpglg2016} was present in the HRS data. To do this I used the available income data from the 2020 wave with the trust measures from the same year.

\par First, I consider the statistical relationship for labor income in the following figures. 

\begin{table}[htbp]\centering
\def\sym#1{\ifmmode^{#1}\else\(^{#1}\)\fi}
\caption{Labor income (2020) on General trust (2020)}
\label{tab:income_trust_general_labor_log}
\begin{tabular*}{0.85\hsize}{@{\hskip\tabcolsep\extracolsep\fill}l*{4}{D{.}{.}{-1}}}
\toprule
                    &\multicolumn{1}{c}{1}&\multicolumn{1}{c}{2}&\multicolumn{1}{c}{3}&\multicolumn{1}{c}{4}\\
\midrule
General trust       &        0.02         &        0.19\sym{**} &        0.01         &        0.07         \\
                    &      (0.03)         &      (0.08)         &      (0.03)         &      (0.08)         \\
(General)$^2$       &                     &       -0.02\sym{**} &                     &       -0.01         \\
                    &                     &      (0.01)         &                     &      (0.01)         \\
Female              &                     &                     &       -0.27\sym{**} &       -0.26\sym{*}  \\
                    &                     &                     &      (0.13)         &      (0.13)         \\
Years of education  &                     &                     &        0.11\sym{***}&        0.10\sym{***}\\
                    &                     &                     &      (0.02)         &      (0.02)         \\
Married             &                     &                     &        0.32\sym{**} &        0.31\sym{**} \\
                    &                     &                     &      (0.13)         &      (0.13)         \\
Born in U.S.        &                     &                     &       -0.11         &       -0.11         \\
                    &                     &                     &      (0.18)         &      (0.18)         \\
In labor force      &                     &                     &        0.67\sym{***}&        0.65\sym{***}\\
                    &                     &                     &      (0.22)         &      (0.22)         \\
NH Black            &                     &                     &       -0.10         &       -0.09         \\
                    &                     &                     &      (0.15)         &      (0.15)         \\
Hispanic            &                     &                     &       -0.21         &       -0.19         \\
                    &                     &                     &      (0.19)         &      (0.19)         \\
NH Other            &                     &                     &       -0.30         &       -0.31         \\
                    &                     &                     &      (0.23)         &      (0.23)         \\
Constant            &       10.23\sym{***}&        9.92\sym{***}&        7.75\sym{***}&        7.59\sym{***}\\
                    &      (0.17)         &      (0.22)         &      (0.49)         &      (0.53)         \\
\midrule
Observations        &      347.00         &      347.00         &      343.00         &      343.00         \\
Adj. R-squared      &       -0.00         &        0.01         &        0.16         &        0.16         \\
Joint test: Trust+Trust² p-value&                     &        0.06         &                     &        0.63         \\
\bottomrule
\multicolumn{5}{l}{\footnotesize Standard errors in parentheses}\\
\multicolumn{5}{l}{\footnotesize Age bins (5-yr) included in columns 3–4.}\\
\multicolumn{5}{l}{\footnotesize \sym{*} \(p<0.10\), \sym{**} \(p<0.05\), \sym{***} \(p<0.01\)}\\
\end{tabular*}
\end{table}


\begin{table}[htbp]\centering
\def\sym#1{\ifmmode^{#1}\else\(^{#1}\)\fi}
\caption{Total income (2020) on General trust (2020), scaled asinh}
\label{tab:income_trust_general_total_ihs}
\begin{tabular*}{0.85\hsize}{@{\hskip\tabcolsep\extracolsep\fill}l*{4}{D{.}{.}{-1}}}
\toprule
                    &\multicolumn{1}{c}{1}&\multicolumn{1}{c}{2}&\multicolumn{1}{c}{3}&\multicolumn{1}{c}{4}\\
\midrule
General trust       &        0.03\sym{***}&        0.19\sym{***}&        0.00         &        0.05\sym{**} \\
                    &      (0.01)         &      (0.02)         &      (0.01)         &      (0.02)         \\
(General)$^2$       &                     &       -0.02\sym{***}&                     &       -0.00\sym{**} \\
                    &                     &      (0.00)         &                     &      (0.00)         \\
Female              &                     &                     &       -0.27\sym{***}&       -0.27\sym{***}\\
                    &                     &                     &      (0.04)         &      (0.04)         \\
Years of education  &                     &                     &        0.07\sym{***}&        0.07\sym{***}\\
                    &                     &                     &      (0.01)         &      (0.01)         \\
Married             &                     &                     &        0.14\sym{***}&        0.14\sym{***}\\
                    &                     &                     &      (0.04)         &      (0.04)         \\
Born in U.S.        &                     &                     &        0.02         &        0.02         \\
                    &                     &                     &      (0.07)         &      (0.07)         \\
In labor force      &                     &                     &        0.50\sym{***}&        0.49\sym{***}\\
                    &                     &                     &      (0.05)         &      (0.05)         \\
NH Black            &                     &                     &       -0.24\sym{***}&       -0.23\sym{***}\\
                    &                     &                     &      (0.05)         &      (0.05)         \\
Hispanic            &                     &                     &       -0.24\sym{***}&       -0.23\sym{***}\\
                    &                     &                     &      (0.07)         &      (0.07)         \\
NH Other            &                     &                     &       -0.27\sym{***}&       -0.27\sym{***}\\
                    &                     &                     &      (0.09)         &      (0.09)         \\
Constant            &        0.83\sym{***}&        0.53\sym{***}&       -0.48\sym{***}&       -0.51\sym{***}\\
                    &      (0.05)         &      (0.06)         &      (0.14)         &      (0.16)         \\
\midrule
Observations        &      900.00         &      900.00         &      890.00         &      890.00         \\
Adj. R-squared      &        0.01         &        0.05         &        0.35         &        0.35         \\
\bottomrule
\multicolumn{5}{l}{\footnotesize Standard errors in parentheses}\\
\multicolumn{5}{l}{\footnotesize Age bins (5-yr) included in columns 3–4.}\\
\multicolumn{5}{l}{\footnotesize \sym{*} \(p<0.10\), \sym{**} \(p<0.05\), \sym{***} \(p<0.01\)}\\
\end{tabular*}
\end{table}


\par The pattern seems to be stronger for total income, based on the figure .

\begin{table}[htbp]\centering
\def\sym#1{\ifmmode^{#1}\else\(^{#1}\)\fi}
\caption{Total income (2020) on General trust (2020)}
\label{tab:income_trust_general_total_log}
\begin{tabular*}{0.85\hsize}{@{\hskip\tabcolsep\extracolsep\fill}l*{4}{D{.}{.}{-1}}}
\toprule
                    &\multicolumn{1}{c}{1}&\multicolumn{1}{c}{2}&\multicolumn{1}{c}{3}&\multicolumn{1}{c}{4}\\
\midrule
General trust       &        0.03\sym{**} &        0.29\sym{***}&        0.00         &        0.08\sym{**} \\
                    &      (0.01)         &      (0.04)         &      (0.01)         &      (0.04)         \\
(General)$^2$       &                     &       -0.03\sym{***}&                     &       -0.01\sym{**} \\
                    &                     &      (0.00)         &                     &      (0.00)         \\
Female              &                     &                     &       -0.40\sym{***}&       -0.39\sym{***}\\
                    &                     &                     &      (0.07)         &      (0.07)         \\
Years of education  &                     &                     &        0.10\sym{***}&        0.09\sym{***}\\
                    &                     &                     &      (0.01)         &      (0.01)         \\
Married             &                     &                     &        0.22\sym{***}&        0.21\sym{***}\\
                    &                     &                     &      (0.07)         &      (0.07)         \\
Born in U.S.        &                     &                     &        0.08         &        0.08         \\
                    &                     &                     &      (0.11)         &      (0.11)         \\
In labor force      &                     &                     &        0.75\sym{***}&        0.74\sym{***}\\
                    &                     &                     &      (0.08)         &      (0.08)         \\
NH Black            &                     &                     &       -0.33\sym{***}&       -0.31\sym{***}\\
                    &                     &                     &      (0.08)         &      (0.08)         \\
Hispanic            &                     &                     &       -0.28\sym{***}&       -0.28\sym{***}\\
                    &                     &                     &      (0.10)         &      (0.11)         \\
NH Other            &                     &                     &       -0.35\sym{**} &       -0.35\sym{**} \\
                    &                     &                     &      (0.14)         &      (0.14)         \\
Constant            &       10.12\sym{***}&        9.63\sym{***}&        8.09\sym{***}&        7.95\sym{***}\\
                    &      (0.09)         &      (0.11)         &      (0.24)         &      (0.25)         \\
\midrule
Observations        &      856.00         &      856.00         &      848.00         &      848.00         \\
Adj. R-squared      &        0.00         &        0.05         &        0.33         &        0.33         \\
\bottomrule
\multicolumn{5}{l}{\footnotesize Standard errors in parentheses}\\
\multicolumn{5}{l}{\footnotesize Age bins (5-yr) included in columns 3–4.}\\
\multicolumn{5}{l}{\footnotesize \sym{*} \(p<0.10\), \sym{**} \(p<0.05\), \sym{***} \(p<0.01\)}\\
\end{tabular*}
\end{table}


\begin{table}[htbp]\centering
\def\sym#1{\ifmmode^{#1}\else\(^{#1}\)\fi}
\caption{Total income (2020) on General trust (2020), scaled asinh}
\label{tab:income_trust_general_total_ihs}
\begin{tabular*}{0.85\hsize}{@{\hskip\tabcolsep\extracolsep\fill}l*{4}{D{.}{.}{-1}}}
\toprule
                    &\multicolumn{1}{c}{1}&\multicolumn{1}{c}{2}&\multicolumn{1}{c}{3}&\multicolumn{1}{c}{4}\\
\midrule
General trust       &        0.03\sym{***}&        0.19\sym{***}&        0.00         &        0.05\sym{**} \\
                    &      (0.01)         &      (0.02)         &      (0.01)         &      (0.02)         \\
(General)$^2$       &                     &       -0.02\sym{***}&                     &       -0.00\sym{**} \\
                    &                     &      (0.00)         &                     &      (0.00)         \\
Female              &                     &                     &       -0.27\sym{***}&       -0.27\sym{***}\\
                    &                     &                     &      (0.04)         &      (0.04)         \\
Years of education  &                     &                     &        0.07\sym{***}&        0.07\sym{***}\\
                    &                     &                     &      (0.01)         &      (0.01)         \\
Married             &                     &                     &        0.14\sym{***}&        0.14\sym{***}\\
                    &                     &                     &      (0.04)         &      (0.04)         \\
Born in U.S.        &                     &                     &        0.02         &        0.02         \\
                    &                     &                     &      (0.07)         &      (0.07)         \\
In labor force      &                     &                     &        0.50\sym{***}&        0.49\sym{***}\\
                    &                     &                     &      (0.05)         &      (0.05)         \\
NH Black            &                     &                     &       -0.24\sym{***}&       -0.23\sym{***}\\
                    &                     &                     &      (0.05)         &      (0.05)         \\
Hispanic            &                     &                     &       -0.24\sym{***}&       -0.23\sym{***}\\
                    &                     &                     &      (0.07)         &      (0.07)         \\
NH Other            &                     &                     &       -0.27\sym{***}&       -0.27\sym{***}\\
                    &                     &                     &      (0.09)         &      (0.09)         \\
Constant            &        0.83\sym{***}&        0.53\sym{***}&       -0.48\sym{***}&       -0.51\sym{***}\\
                    &      (0.05)         &      (0.06)         &      (0.14)         &      (0.16)         \\
\midrule
Observations        &      900.00         &      900.00         &      890.00         &      890.00         \\
Adj. R-squared      &        0.01         &        0.05         &        0.35         &        0.35         \\
\bottomrule
\multicolumn{5}{l}{\footnotesize Standard errors in parentheses}\\
\multicolumn{5}{l}{\footnotesize Age bins (5-yr) included in columns 3–4.}\\
\multicolumn{5}{l}{\footnotesize \sym{*} \(p<0.10\), \sym{**} \(p<0.05\), \sym{***} \(p<0.01\)}\\
\end{tabular*}
\end{table}


\subsection{Returns and trust}

\par With the hump-shape relationship between earnings and trust present in the HRS, I wanted to see if a similar relationship held for returns.

\par The smallest portfolio composition is of core assets only. The regression results for raw and winsorized annual returns for 2022 are given in figures and .

\begin{table}[htbp]\centering
\def\sym#1{\ifmmode^{#1}\else\(^{#1}\)\fi}
\caption{2022 Returns to core (2022) on General trust (2020) (5% winsorized)}
\label{tab:returns_r1_trust_general_win}
\begin{tabular*}{0.85\hsize}{@{\hskip\tabcolsep\extracolsep\fill}l*{4}{D{{.}}{{.}}{{-1}}}}
\toprule
                    &\multicolumn{1}{c}{1}&\multicolumn{1}{c}{2}&\multicolumn{1}{c}{3}&\multicolumn{1}{c}{4}\\
\midrule
General trust       &        0.02         &        0.16\sym{**} &        0.02         &        0.14\sym{*}  \\
                    &      (0.02)         &      (0.07)         &      (0.02)         &      (0.07)         \\
(General)$^2$       &                     &       -0.01\sym{**} &                     &       -0.01\sym{*}  \\
                    &                     &      (0.01)         &                     &      (0.01)         \\
Female              &                     &                     &       -0.08         &       -0.06         \\
                    &                     &                     &      (0.09)         &      (0.10)         \\
Years of education  &                     &                     &        0.05\sym{**} &        0.04\sym{**} \\
                    &                     &                     &      (0.02)         &      (0.02)         \\
Married             &                     &                     &        0.17         &        0.17         \\
                    &                     &                     &      (0.10)         &      (0.10)         \\
Born in U.S.        &                     &                     &       -0.00         &        0.00         \\
                    &                     &                     &      (0.14)         &      (0.14)         \\
NH Black            &                     &                     &       -0.30\sym{**} &       -0.28\sym{**} \\
                    &                     &                     &      (0.13)         &      (0.13)         \\
Hispanic            &                     &                     &       -0.50\sym{***}&       -0.49\sym{***}\\
                    &                     &                     &      (0.19)         &      (0.19)         \\
NH Other            &                     &                     &       -0.21         &       -0.20         \\
                    &                     &                     &      (0.21)         &      (0.21)         \\
In labor force      &                     &                     &        0.14         &        0.14         \\
                    &                     &                     &      (0.11)         &      (0.11)         \\
\_cons              &        0.15         &       -0.14         &        0.97\sym{**} &        0.73         \\
                    &      (0.15)         &      (0.21)         &      (0.43)         &      (0.45)         \\
\midrule
Observations        &      442.00         &      442.00         &      438.00         &      438.00         \\
Adj. R-squared      &        0.00         &        0.01         &        0.09         &        0.10         \\
Joint test: Trust p-value&                     &        0.08         &                     &        0.18         \\
\bottomrule
\multicolumn{5}{l}{\footnotesize Standard errors in parentheses}\\
\multicolumn{5}{l}{\footnotesize Robust standard errors in parentheses. Age bins (5-yr) and wealth deciles included in columns 3–4.}\\
\multicolumn{5}{l}{\footnotesize \sym{*} \(p<0.10\), \sym{**} \(p<0.05\), \sym{***} \(p<0.01\)}\\
\end{tabular*}
\end{table}

\FloatBarrier

\begin{table}[htbp]\centering
\def\sym#1{\ifmmode^{#1}\else\(^{#1}\)\fi}
\caption{2022 Returns to core (2022) on General trust (2020) (raw)}
\label{tab:returns_r1_trust_general}
\begin{tabular*}{0.85\hsize}{@{\hskip\tabcolsep\extracolsep\fill}l*{4}{D{{.}}{{.}}{{-1}}}}
\toprule
                    &\multicolumn{1}{c}{1}&\multicolumn{1}{c}{2}&\multicolumn{1}{c}{3}&\multicolumn{1}{c}{4}\\
\midrule
General trust       &        0.03         &        0.22\sym{*}  &       -0.01         &        0.18         \\
                    &      (0.04)         &      (0.12)         &      (0.04)         &      (0.13)         \\
(General)$^2$       &                     &       -0.02\sym{*}  &                     &       -0.02         \\
                    &                     &      (0.01)         &                     &      (0.01)         \\
Female              &                     &                     &        0.23         &        0.27         \\
                    &                     &                     &      (0.21)         &      (0.21)         \\
Years of education  &                     &                     &        0.16\sym{***}&        0.15\sym{***}\\
                    &                     &                     &      (0.05)         &      (0.05)         \\
Married             &                     &                     &        0.55\sym{**} &        0.55\sym{**} \\
                    &                     &                     &      (0.24)         &      (0.24)         \\
Born in U.S.        &                     &                     &        0.18         &        0.18         \\
                    &                     &                     &      (0.27)         &      (0.27)         \\
NH Black            &                     &                     &       -0.50\sym{*}  &       -0.48\sym{*}  \\
                    &                     &                     &      (0.27)         &      (0.27)         \\
Hispanic            &                     &                     &       -0.32         &       -0.31         \\
                    &                     &                     &      (0.41)         &      (0.41)         \\
NH Other            &                     &                     &       -0.16         &       -0.16         \\
                    &                     &                     &      (0.52)         &      (0.52)         \\
In labor force      &                     &                     &        0.30         &        0.31         \\
                    &                     &                     &      (0.22)         &      (0.22)         \\
\_cons              &        0.38         &       -0.06         &       -1.76         &       -2.16         \\
                    &      (0.23)         &      (0.37)         &      (1.48)         &      (1.58)         \\
\midrule
Observations        &      442.00         &      442.00         &      438.00         &      438.00         \\
Adj. R-squared      &       -0.00         &        0.00         &        0.09         &        0.10         \\
Joint test: Trust p-value&                     &        0.16         &                     &        0.25         \\
\bottomrule
\multicolumn{5}{l}{\footnotesize Standard errors in parentheses}\\
\multicolumn{5}{l}{\footnotesize Robust standard errors in parentheses. Age bins (5-yr) and wealth deciles included in columns 3–4.}\\
\multicolumn{5}{l}{\footnotesize \sym{*} \(p<0.10\), \sym{**} \(p<0.05\), \sym{***} \(p<0.01\)}\\
\end{tabular*}
\end{table}

\FloatBarrier

\par The next portfolio composition includes retirement assets along with those core assets. The regression results for raw and winsorized annual returns for 2022 on these portfolios are given in figures and .

\begin{table}[htbp]\centering
\def\sym#1{\ifmmode^{#1}\else\(^{#1}\)\fi}
\caption{2022 Core+res return (2022) on General trust (2020) (5 percent winsorized)}
\label{tab:returns_r4_trust_general_win}
\begin{tabular*}{0.85\hsize}{@{\hskip\tabcolsep\extracolsep\fill}l*{4}{D{.}{.}{-1}}}
\toprule
                    &\multicolumn{1}{c}{1}&\multicolumn{1}{c}{2}&\multicolumn{1}{c}{3}&\multicolumn{1}{c}{4}\\
\midrule
General trust       &        0.01         &        0.02         &        0.03         &        0.02         \\
                    &      (0.02)         &      (0.04)         &      (0.02)         &      (0.05)         \\
(General)$^2$       &                     &       -0.00         &                     &        0.00         \\
                    &                     &      (0.00)         &                     &      (0.00)         \\
Female              &                     &                     &       -0.10\sym{*}  &       -0.10\sym{*}  \\
                    &                     &                     &      (0.06)         &      (0.06)         \\
Years of education  &                     &                     &        0.02         &        0.02         \\
                    &                     &                     &      (0.01)         &      (0.01)         \\
Married             &                     &                     &        0.08         &        0.08         \\
                    &                     &                     &      (0.06)         &      (0.06)         \\
Born in U.S.        &                     &                     &        0.12         &        0.12         \\
                    &                     &                     &      (0.08)         &      (0.08)         \\
NH Black            &                     &                     &       -0.09         &       -0.09         \\
                    &                     &                     &      (0.08)         &      (0.08)         \\
Hispanic            &                     &                     &       -0.33\sym{*}  &       -0.33\sym{*}  \\
                    &                     &                     &      (0.18)         &      (0.18)         \\
NH Other            &                     &                     &        0.01         &        0.01         \\
                    &                     &                     &      (0.11)         &      (0.11)         \\
In labor force      &                     &                     &        0.05         &        0.05         \\
                    &                     &                     &      (0.06)         &      (0.06)         \\
\_cons              &        0.10         &        0.08         &       -0.28         &       -0.26         \\
                    &      (0.10)         &      (0.12)         &      (0.32)         &      (0.32)         \\
\midrule
Observations        &      210.00         &      210.00         &      209.00         &      209.00         \\
Adj. R-squared      &       -0.00         &       -0.01         &        0.09         &        0.09         \\
\bottomrule
\multicolumn{5}{l}{\footnotesize Standard errors in parentheses}\\
\multicolumn{5}{l}{\footnotesize Robust standard errors in parentheses. Age bins (5-yr) and wealth deciles included in columns 3–4.}\\
\multicolumn{5}{l}{\footnotesize \sym{*} \(p<0.10\), \sym{**} \(p<0.05\), \sym{***} \(p<0.01\)}\\
\end{tabular*}
\end{table}

\FloatBarrier

\begin{table}[htbp]\centering
\def\sym#1{\ifmmode^{#1}\else\(^{#1}\)\fi}
\caption{2022 Returns to core+IRA (2022) on General trust (2020) (raw)}
\label{tab:returns_r4_trust_general}
\begin{tabular*}{0.85\hsize}{@{\hskip\tabcolsep\extracolsep\fill}l*{4}{D{.}{.}{-1}}}
\toprule
                    &\multicolumn{1}{c}{1}&\multicolumn{1}{c}{2}&\multicolumn{1}{c}{3}&\multicolumn{1}{c}{4}\\
\midrule
General trust       &        0.02         &        0.04         &        0.03         &        0.04         \\
                    &      (0.02)         &      (0.05)         &      (0.02)         &      (0.05)         \\
(General)$^2$       &                     &       -0.00         &                     &       -0.00         \\
                    &                     &      (0.00)         &                     &      (0.00)         \\
Female              &                     &                     &       -0.11         &       -0.11         \\
                    &                     &                     &      (0.07)         &      (0.07)         \\
Years of education  &                     &                     &        0.03         &        0.03         \\
                    &                     &                     &      (0.02)         &      (0.02)         \\
Married             &                     &                     &        0.06         &        0.06         \\
                    &                     &                     &      (0.09)         &      (0.09)         \\
Born in U.S.        &                     &                     &        0.14         &        0.14         \\
                    &                     &                     &      (0.08)         &      (0.08)         \\
NH Black            &                     &                     &       -0.11         &       -0.11         \\
                    &                     &                     &      (0.09)         &      (0.09)         \\
Hispanic            &                     &                     &       -0.39\sym{**} &       -0.40\sym{*}  \\
                    &                     &                     &      (0.20)         &      (0.20)         \\
NH Other            &                     &                     &       -0.03         &       -0.03         \\
                    &                     &                     &      (0.13)         &      (0.13)         \\
In labor force      &                     &                     &        0.04         &        0.03         \\
                    &                     &                     &      (0.07)         &      (0.07)         \\
\_cons              &        0.09         &        0.02         &       -0.28         &       -0.30         \\
                    &      (0.12)         &      (0.15)         &      (0.43)         &      (0.44)         \\
\midrule
Observations        &      210.00         &      210.00         &      209.00         &      209.00         \\
Adj. R-squared      &       -0.00         &       -0.01         &        0.07         &        0.07         \\
\bottomrule
\multicolumn{5}{l}{\footnotesize Standard errors in parentheses}\\
\multicolumn{5}{l}{\footnotesize Robust standard errors in parentheses. Age bins (5-yr) and wealth deciles included in columns 3–4.}\\
\multicolumn{5}{l}{\footnotesize \sym{*} \(p<0.10\), \sym{**} \(p<0.05\), \sym{***} \(p<0.01\)}\\
\end{tabular*}
\end{table}

\FloatBarrier


\par The largest portfolio composition is on net wealth, which includes the previous assets, residential assets, along with other smaller categories (and debt variables). The regression results for raw and winsorized annual returns to net wealth for 2022 are given in figures and .

\begin{table}[htbp]\centering
\def\sym#1{\ifmmode^{#1}\else\(^{#1}\)\fi}
\caption{2022 Returns to net wealth (2022) on General trust (2020) (5 percent winsorized)}
\label{tab:returns_r5_trust_general_win}
\begin{tabular*}{0.85\hsize}{@{\hskip\tabcolsep\extracolsep\fill}l*{4}{D{.}{.}{-1}}}
\toprule
                    &\multicolumn{1}{c}{1}&\multicolumn{1}{c}{2}&\multicolumn{1}{c}{3}&\multicolumn{1}{c}{4}\\
\midrule
General trust       &        0.01         &        0.06\sym{**} &        0.01         &        0.03         \\
                    &      (0.01)         &      (0.02)         &      (0.01)         &      (0.03)         \\
(General)$^2$       &                     &       -0.00\sym{*}  &                     &       -0.00         \\
                    &                     &      (0.00)         &                     &      (0.00)         \\
Female              &                     &                     &       -0.10\sym{**} &       -0.09\sym{**} \\
                    &                     &                     &      (0.04)         &      (0.04)         \\
Years of education  &                     &                     &        0.03\sym{***}&        0.02\sym{***}\\
                    &                     &                     &      (0.01)         &      (0.01)         \\
Married             &                     &                     &        0.08         &        0.08         \\
                    &                     &                     &      (0.05)         &      (0.05)         \\
Born in U.S.        &                     &                     &        0.03         &        0.03         \\
                    &                     &                     &      (0.05)         &      (0.05)         \\
NH Black            &                     &                     &       -0.16\sym{**} &       -0.15\sym{**} \\
                    &                     &                     &      (0.06)         &      (0.07)         \\
Hispanic            &                     &                     &       -0.13\sym{*}  &       -0.13\sym{*}  \\
                    &                     &                     &      (0.07)         &      (0.07)         \\
NH Other            &                     &                     &       -0.05         &       -0.05         \\
                    &                     &                     &      (0.12)         &      (0.12)         \\
In labor force      &                     &                     &        0.13\sym{**} &        0.13\sym{**} \\
                    &                     &                     &      (0.05)         &      (0.05)         \\
\_cons              &        0.24\sym{***}&        0.13\sym{**} &       -0.49\sym{**} &       -0.47\sym{**} \\
                    &      (0.06)         &      (0.07)         &      (0.20)         &      (0.20)         \\
\midrule
Observations        &      497.00         &      497.00         &      406.00         &      406.00         \\
Adj. R-squared      &        0.00         &        0.01         &        0.23         &        0.23         \\
Joint test: Trust p-value&                     &        0.03         &                     &        0.54         \\
\bottomrule
\multicolumn{5}{l}{\footnotesize Standard errors in parentheses}\\
\multicolumn{5}{l}{\footnotesize Robust standard errors in parentheses. Age bins (5-yr) and wealth deciles included in columns 3–4.}\\
\multicolumn{5}{l}{\footnotesize \sym{*} \(p<0.10\), \sym{**} \(p<0.05\), \sym{***} \(p<0.01\)}\\
\end{tabular*}
\end{table}

\FloatBarrier

\begin{table}[htbp]\centering
\def\sym#1{\ifmmode^{#1}\else\(^{#1}\)\fi}
\caption{2022 Returns to net wealth (2022) on General trust (2020) (raw)}
\label{tab:returns_r5_trust_general}
\begin{tabular*}{0.85\hsize}{@{\hskip\tabcolsep\extracolsep\fill}l*{4}{D{.}{.}{-1}}}
\toprule
                    &\multicolumn{1}{c}{1}&\multicolumn{1}{c}{2}&\multicolumn{1}{c}{3}&\multicolumn{1}{c}{4}\\
\midrule
General trust       &        0.01         &        0.10\sym{***}&        0.01         &        0.07\sym{*}  \\
                    &      (0.01)         &      (0.03)         &      (0.01)         &      (0.04)         \\
(General)$^2$       &                     &       -0.01\sym{**} &                     &       -0.01         \\
                    &                     &      (0.00)         &                     &      (0.00)         \\
Female              &                     &                     &       -0.07         &       -0.06         \\
                    &                     &                     &      (0.07)         &      (0.07)         \\
Years of education  &                     &                     &        0.03\sym{**} &        0.03\sym{*}  \\
                    &                     &                     &      (0.02)         &      (0.02)         \\
Married             &                     &                     &        0.09         &        0.09         \\
                    &                     &                     &      (0.08)         &      (0.08)         \\
Born in U.S.        &                     &                     &        0.07         &        0.07         \\
                    &                     &                     &      (0.08)         &      (0.08)         \\
NH Black            &                     &                     &       -0.15         &       -0.13         \\
                    &                     &                     &      (0.10)         &      (0.10)         \\
Hispanic            &                     &                     &       -0.09         &       -0.09         \\
                    &                     &                     &      (0.14)         &      (0.14)         \\
NH Other            &                     &                     &       -0.10         &       -0.08         \\
                    &                     &                     &      (0.14)         &      (0.15)         \\
In labor force      &                     &                     &        0.18\sym{***}&        0.18\sym{***}\\
                    &                     &                     &      (0.07)         &      (0.07)         \\
\_cons              &        0.30\sym{***}&        0.11         &       -0.53         &       -0.49         \\
                    &      (0.07)         &      (0.08)         &      (0.45)         &      (0.44)         \\
\midrule
Observations        &      497.00         &      497.00         &      406.00         &      406.00         \\
Adj. R-squared      &       -0.00         &        0.01         &        0.17         &        0.18         \\
Joint test: Trust p-value&                     &        0.01         &                     &        0.22         \\
\bottomrule
\multicolumn{5}{l}{\footnotesize Standard errors in parentheses}\\
\multicolumn{5}{l}{\footnotesize Robust standard errors in parentheses. Age bins (5-yr) and wealth deciles included in columns 3–4.}\\
\multicolumn{5}{l}{\footnotesize \sym{*} \(p<0.10\), \sym{**} \(p<0.05\), \sym{***} \(p<0.01\)}\\
\end{tabular*}
\end{table}

\FloatBarrier


\subsection{Returns from 2002-2022}

\par Next, I wanted to compute returns across several years and attempt to describe the persistent component of returns that \cite{aflgdmlp20} point to as heterogeneity across individuals. I also use three similar specifications. First, a pooled OLS regression with a baseline set of controls. Second, a similar pooled regression with aimed at controlling for risk exposure. Third, a panel regression with individual fixed effects and year dummies.

\section{The pooled regression with constant trust}

\par After looking at those specifications for the returns regressions, I wanted to see if the trust measure was significant in the pooled setting. 

\subsection{Trust and average income}

\subsection{Trust and average returns}

\subsection{Estimated fixed effects for returns}